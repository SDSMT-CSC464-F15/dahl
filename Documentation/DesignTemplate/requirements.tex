% !TEX root = SystemTemplate.tex
\chapter{User Stories,  Requirements,Backlog and Deliverables}
\section{Overview}


This documents  will cover the Dahl's Vurtial Reality Prototype. The document will give an insight to different aspects to the project. It is cropped into sections to focus on the projects design, unit testing, development environment, documentation, setup, and the customers. Through these sections it allows displays the structure the project has and the needs that it fufills.

%The overview should take the form of an executive summary.  Give the reader a feel 
%for the purpose of the document, what is contained in the document, and an idea 
%of the purpose for the system or product. 

% The user stories 
%are provided by the stakeholders.  You will create the backlogs and the requirements, and document here.  
%This chapter should contain 
%details about each of the requirements and how the requirements are or will be 
%satisfied in the design and implementation of the system.

%Below:   list, describe, and define the requirements in this chapter.  
%There could be any number of sub-sections to help provide the necessary level of 
%detail. 



\subsection{Scope}


This document will cover stakeholder information, 
initial user stories, system requirements, proof of concept results, 
and various research. 



\subsection{Purpose of the System}
The purpose of the Virual Reality Gallary is to allow other to experience the Dahl Art
Mueseum without going to the physical location, as well as preserving select gallaries.  


\section{ Stakeholder Information}


This section would provide the basic description of all of the stakeholders for 
the project.  Who has an interest in the successful and/or unsuccessful completion 
of this project? 


\subsection{Customer or End User (Product Owner)}
Who?  What role will they play in the project?  Will this person or group manage 
and prioritize the product backlog?  Who will they interact with on the team to 
drive product backlog priorities if not done directly? 

\subsection{Management or Instructor (Scrum Master)}
Who?  What role will they play in the project?  Will the Scrum Master drive the 
Sprint Meetings? 


\subsection{Investors}
There are no current investors in this project


\subsection{Developers --Testers}
The developers for this project consist of Alex Nienhueser and Mack Smith. They are both Computer 
Science Majors from SDSMT.

%Who?  Is there a defined project manager, developer, tester, designer, architect, 
%etc.? 


\section{Business Need}
The Virtual Reality Gallery will fufill the need of portability. An art museum such as the Dahl has a difficult time bringing art to those own can't come to their premises. With this project the Dahl will be able to take an entire gallery outside of their current location.

%Use this section to define what business need exist and how this software will 
%meet and/or exceed that business need.   

\section{Requirements and Design Constraints}
The constrants of this project lays only in obtaining a finished product. 

%Use this section to discuss what requirements exist that deal with meeting the 
%business need.  These requirements might equate to design constraints which can 
%take the form of system, network, and/or user constraints.  Examples:  Windows 
%Server only, iOS only, slow network constraints, or no offline, local storage capabilities. 


\subsection{System  Requirements}
The minimum system requirements for this project will be:

NVIDIA GTX 970 / AMD 290 equivalent or greater with Intel i5-4590 processor equivalent or greater. 8GB+ RAM.


%What are they?  How will they impact the potential design?  Are there alternatives? 


\subsection{Network Requirements}
There will be no need for network connection for this project. Network connectivity will not 
affect the Virutal Reality Gallery.


\subsection{Development Environment Requirements}
The Develope requires windows 7 or newer.

%What are they?  Is the system supposed to be cross-platform? 


\subsection{Project  Management Methodology}
The stakeholders might restrict how the project implementation will be managed. 
 There may be constraints on when design meetings will take place.  There might 
be restrictions on how often progress reports need to be provided and to whom. 
 
\begin{itemize}
\item What system will be used to keep track of the backlogs and sprint status?
\item Will all parties have access to the Sprint and Product Backlogs?
\item How many Sprints will encompass this particular project?
\item How long are the Sprint Cycles?
\item Are there restrictions on source control? 
\end{itemize}

\section{User Stories}
%This section can really be seen as the guts of the document.  This section should 
%be the result of discussions with the stakeholders with regard to the actual functional 
%requirements of the software.  It is the user stories that will be used in the 
%work breakdown structure to build tasks to fill the product backlog for implementation 
%through the sprints.

%This section should contain sub-sections to define and potentially provide a breakdown 
%of larger user stories into smaller user stories. 



\subsection{User Story \#1}
Does the first user story need some division into smaller, consumable parts by 
the reader?  This does not need to go to the level of actual task definition and 
may not be required. 

\subsection{User Story \#2} 
As a user I will be able to select between a guided tour and user-controlled experience the latter
will be controlled using a handheld game controller.

\subsection{User Story \#3} 
This product should be able to be transported to make it able to take to other locations and 
allow people who might otherwise be able to experience the Dahl.

\subsection{User Story \#4} 
As a user I will be able to view an art piece and have a description text box appear on the side of
the piece with interpretive text about the piece in view.  There should also be the ability to add 
a short video to play on specific exhibits to add further description for the user.

\subsubsection{User Story \#4 Breakdown}
Text should be provided by the Dahl to ensure customer satisfaction.


\section{Research or Proof of Concept Results}
This section is reserved for the discussion centered on any research that needed 
to take place before full system design.  The research efforts may have led to 
the need to actually provide a proof of concept for approval by the stakeholders. 
 The proof of concept might even go to the extent of a user interface design or 
mockups. 


\section{Supporting Material}


This document might contain references or supporting material which should be documented 
and discussed  either here if appropriate or more often in the appendices at the end.  This material may have been provided by the stakeholders  
or it may be material garnered from research tasks.

